\documentclass[11pt]{article}

% ---------- Packages ----------
\usepackage[margin=1in]{geometry}
\usepackage{amsmath, amssymb, amsfonts}
\usepackage{physics}
\usepackage{mathtools}
\usepackage{hyperref}
\usepackage{graphicx}
\usepackage{booktabs}
\usepackage{array}
\usepackage{xcolor}
\usepackage{authblk}
\usepackage{enumitem}

% Circuits
\usepackage{quantikz}

% Code snippets
\usepackage{listings}
\lstset{
  basicstyle=\ttfamily\small,
  breaklines=true,
  frame=single,
  columns=fullflexible,
  keywordstyle=\color{blue},
  commentstyle=\color{gray},
  stringstyle=\color{teal}
}

% ---------- Title ----------
\title{\textbf{Clifford+T Compilation of Few-Qubit Unitaries for Fault-Tolerant Cost Reduction}\\
\large iQuHACK 2026 Superquantum Challenge (Tasks 1--6, 8--9)}
\author[1]{Pavitra Bhargavi Allamaraju}
\author[2]{Ryan Ma}
\affil[1]{University of British Columbia}
\affil[2]{University of Waterloo}
\date{}

\begin{document}
\maketitle

\begin{abstract}
Fault-tolerant quantum computation imposes significant overhead on non-Clifford operations, particularly the $T$ gate.
This work documents a structured compilation workflow for several 2-qubit target unitaries from the iQuHACK 2026 Superquantum challenge,
producing circuits over the restricted gate set $\{H, T, T^\dagger, \mathrm{CNOT}\}$.
We present exact constructions for controlled-Pauli and structured unitaries, identify a key algebraic simplification linking the isotropic Heisenberg evolution to $\mathrm{SWAP}$,
and provide baseline approximations for Hamiltonian exponentials via standard phase-gadget techniques and first-order product formulas.
We report concise circuit templates and OpenQASM excerpts suitable for submission and subsequent optimization.
\end{abstract}

\section{Introduction}
Fault-tolerant quantum computing architectures typically realize Clifford gates at comparatively low cost, while non-Clifford gates---notably the $T$ gate---dominate resource estimates.
Consequently, compiling target unitaries into Clifford+$T$ form while minimizing $T$-count is a central practical problem.

In the iQuHACK 2026 Superquantum challenge, each submission is an OpenQASM circuit constrained to the basis
$\{H, T, T^\dagger, \mathrm{CNOT}\}$, and is evaluated by (i) operator-norm distance to the target unitary up to global phase and (ii) $T$-count.
This manuscript describes solutions for Tasks 1--6 and 8--9, emphasizing reusable structure: controlled-phase decompositions, Pauli-string exponentials via phase gadgets,
basis-change reductions, and one-step product-formula approximations for non-commuting Hamiltonians.

Tasks 7, 10, and 11 involve generic synthesis and/or multi-qubit diagonal optimization and are excluded from this document.

\section{Background and Notation}
\subsection{Gate set and cost metric}
We compile all circuits into the discrete universal basis
\[
\mathcal{G}=\{H,\; T,\; T^\dagger,\; \mathrm{CNOT}\}.
\]
We use $T$-count as the primary proxy for fault-tolerant cost.

\subsection{Pauli operators and exponentials}
Let $X,Y,Z$ denote Pauli matrices. For any Hermitian operator $M$ with $M^2=I$,
\begin{equation}
e^{i\theta M} = \cos\theta\, I + i\sin\theta\, M.
\end{equation}
For Pauli strings such as $Z\otimes Z$, exponentials can be implemented using parity computation and a single-qubit $Z$-rotation.

\subsection{Phase gadgets for $e^{i\theta Z\otimes Z}$}
A standard identity (up to global phase) is
\begin{equation}
e^{i\theta (Z\otimes Z)} \equiv \mathrm{CNOT}\cdot (I\otimes R_z(2\theta))\cdot \mathrm{CNOT}.
\label{eq:zz-gadget}
\end{equation}
Because $R_z(\varphi)$ is not native in $\mathcal{G}$, we approximate it with Clifford+$T$ sequences when $\varphi$ is not a Clifford angle.


\section{Methods Overview}
Across multiple tasks, we repeatedly apply:
\begin{enumerate}[leftmargin=2em]
\item \textbf{Algebraic reduction:} rewrite the target unitary using known identities (e.g., $Y=SXS^\dagger$, $\mathrm{SWAP}=\tfrac12(I+XX+YY+ZZ)$).
\item \textbf{Basis changes:} conjugate Pauli strings into $Z$-type operators using Clifford gates (typically Hadamards).
\item \textbf{Phase gadgets:} implement $e^{i\theta Z\otimes Z}$ using Eq.~\eqref{eq:zz-gadget}.
\item \textbf{Approximate synthesis:} approximate $R_z(\varphi)$ using short Clifford+$T$ sequences for baseline submissions.
\end{enumerate}
The resulting circuits can be further optimized by dedicated synthesis/optimization tools; here we document compact baseline constructions and exact solutions where available.


\section{Task 1: Controlled-$Y$}
\subsection{Identity}
Using $Y=SXS^\dagger$ and $S=T^2$, we obtain an exact Clifford+$T$ implementation of controlled-$Y$:
\[
\mathrm{CY} = (I\otimes S)\cdot \mathrm{CNOT}\cdot (I\otimes S^\dagger),
\quad S=T^2.
\]

\subsection{Circuit (quantikz)}
\[
\begin{quantikz}
\lstick{$q_0$} & \ctrl{1} & \qw \\
\lstick{$q_1$} & \gate{T^2} & \targ{} & \gate{(T^\dagger)^2} & \qw
\end{quantikz}
\]

\subsection{OpenQASM excerpt}
\begin{lstlisting}
// task01_controlledY.qasm
t q[1];
t q[1];
cx q[0], q[1];
tdg q[1];
tdg q[1];
\end{lstlisting}


\section{Task 2: Controlled-$R_y(\pi/7)$}
\subsection{Decomposition}
A standard controlled-rotation decomposition is
\begin{equation}
\mathrm{CR}_y(\theta)=
(I\otimes R_y(\tfrac{\theta}{2}))\;\mathrm{CNOT}\;
(I\otimes R_y(-\tfrac{\theta}{2}))\;\mathrm{CNOT},
\label{eq:cry}
\end{equation}
and single-qubit rotations satisfy
\begin{equation}
R_y(\alpha)=H\,R_z(\alpha)\,H.
\label{eq:ry-rz}
\end{equation}
Here $\theta=\pi/7$, so we require $R_z(\pm \pi/14)$ on the target qubit, approximated in Clifford+$T$.

\subsection{Circuit template}
\[
\begin{quantikz}
\lstick{$q_0$} & \ctrl{1} & \ctrl{1} & \qw \\
\lstick{$q_1$} & \gate{R_y(\theta/2)} & \targ{} & \gate{R_y(-\theta/2)} & \targ{} & \qw
\end{quantikz}
\]

\subsection{Baseline OpenQASM structure}
We implement $R_y(\pm \theta/2)$ via Eq.~\eqref{eq:ry-rz} with a short Clifford+$T$ approximation of $R_z(\pm\pi/14)$.
\begin{lstlisting}
// task02_controlledRy_pi7.qasm (baseline structure)
h q[1];
t q[1];      // placeholder approx for Rz(+pi/14)
h q[1];

cx q[0], q[1];

h q[1];
tdg q[1];    // placeholder approx for Rz(-pi/14)
h q[1];

cx q[0], q[1];
\end{lstlisting}


\section{Task 3: $e^{i(\pi/7)\, Z\otimes Z}$}
\subsection{Phase gadget}
We use the two-qubit $ZZ$ phase gadget (Eq.~\eqref{eq:zz-gadget}) with $\theta=\pi/7$:
\[
e^{i(\pi/7) Z\otimes Z}\equiv
\mathrm{CNOT}\cdot (I\otimes R_z(2\pi/7))\cdot \mathrm{CNOT}
\quad (\text{global phase ignored}).
\]

\subsection{Circuit (quantikz)}
\[
\begin{quantikz}
\lstick{$q_0$} & \ctrl{1} & \ctrl{1} & \qw \\
\lstick{$q_1$} & \targ{} & \gate{R_z(2\pi/7)} & \targ{} & \qw
\end{quantikz}
\]

\subsection{Baseline OpenQASM structure}
\begin{lstlisting}
// task03_exp_i_pi7_ZZ.qasm (baseline)
cx q[0], q[1];
// approximate Rz(2*pi/7) using a short Clifford+T sequence (placeholder)
t q[1];
t q[1];
t q[1];
t q[1];
cx q[0], q[1];
\end{lstlisting}


\section{Task 4: $e^{i(\pi/7)(XX+YY)}$}
\subsection{Basis reduction}
We reduce $(XX+YY)$ to a $ZZ$-type interaction by a Clifford basis change and then apply a $ZZ$ phase gadget.
A convenient template is:
\[
e^{i\theta(XX+YY)} = (H\otimes H)\;\Big(e^{i\theta(Z\otimes Z)}\Big)\;(H\otimes H),
\quad \theta=\pi/7,
\]
where $e^{i\theta(Z\otimes Z)}$ is implemented using Eq.~\eqref{eq:zz-gadget}.

\subsection{Baseline OpenQASM}
\begin{lstlisting}
// task04_exp_i_pi7_XX_YY.qasm (baseline)
h q[0];
h q[1];

cx q[0], q[1];
t q[1];
t q[1];
t q[1];
t q[1];
cx q[0], q[1];

h q[0];
h q[1];
\end{lstlisting}


\section{Task 5: $e^{i(\pi/4)(XX+YY+ZZ)}$}
\subsection{Key structure: SWAP equivalence}
The SWAP operator satisfies
\begin{equation}
\mathrm{SWAP}=\frac12\left(I+XX+YY+ZZ\right)
\quad \Rightarrow \quad
XX+YY+ZZ = 2\,\mathrm{SWAP}-I.
\label{eq:swap-identity}
\end{equation}
Thus,
\[
e^{i\frac{\pi}{4}(XX+YY+ZZ)} = e^{i\frac{\pi}{4}(2\,\mathrm{SWAP}-I)}
= e^{-i\pi/4}\, e^{i\frac{\pi}{2}\mathrm{SWAP}}.
\]
Since $\mathrm{SWAP}^2=I$, we have $e^{i(\pi/2)\mathrm{SWAP}} = i\,\mathrm{SWAP}$, hence the target unitary equals $\mathrm{SWAP}$ up to a global phase (ignored by the grader).

\subsection{Exact circuit}
\begin{lstlisting}
// task05_exp_i_pi4_XX_YY_ZZ.qasm (exact)
cx q[0], q[1];
cx q[1], q[0];
cx q[0], q[1];
\end{lstlisting}


\section{Task 6: $e^{i(\pi/7)(XX+ZI+IZ)}$}
\subsection{Non-commuting Hamiltonian and baseline product formula}
The terms $XX$, $ZI$, and $IZ$ do not all commute. A baseline approach uses a first-order Lie--Trotter product:
\begin{equation}
e^{i\theta(XX+ZI+IZ)} \approx e^{i\theta XX}\, e^{i\theta ZI}\, e^{i\theta IZ},
\quad \theta=\pi/7.
\label{eq:trotter1}
\end{equation}
The single-qubit factors satisfy
\[
e^{i\theta ZI} = R_z(2\theta)\otimes I,
\qquad
e^{i\theta IZ} = I\otimes R_z(2\theta),
\]
and the two-qubit factor is reduced via basis change:
\[
e^{i\theta XX} = (H\otimes H)\; e^{i\theta ZZ}\; (H\otimes H).
\]

\subsection{Baseline OpenQASM}
\begin{lstlisting}
// task06_exp_i_pi7_XX_ZI_IZ.qasm (baseline structure)
h q[0];
h q[1];

cx q[0], q[1];
t q[1];          // placeholder for Rz(2*pi/7)
cx q[0], q[1];

h q[0];
h q[1];

t q[0];          // placeholder for Rz(2*pi/7) on q0
t q[1];          // placeholder for Rz(2*pi/7) on q1
\end{lstlisting}


\section{Task 8: Structured Unitary 1 (2-qubit QFT)}
\subsection{Structure}
The given matrix equals the 2-qubit Quantum Fourier Transform $\mathrm{QFT}_2$ (up to global phase). A compact factorization is
\[
\mathrm{QFT}_2 = (H\otimes I)\cdot \mathrm{CS}\cdot (I\otimes H),
\]
where $\mathrm{CS}$ is controlled-$S$ and $S=T^2$.

\subsection{Controlled-$S$ using Clifford+$T$}
Since $S=T^2$, we implement $\mathrm{CS}$ with phase kickback and CNOTs (exact over $\mathcal{G}$).

\subsection{OpenQASM (exact)}
\begin{lstlisting}
// task08_structured_unitary_qft2.qasm
h q[0];

// controlled-S (control q0, target q1)
t q[1];
cx q[0], q[1];
tdg q[1];
cx q[0], q[1];
t q[0];

h q[1];
\end{lstlisting}


\section{Task 9: Structured Unitary 2}
\subsection{Structure}
The matrix is block-structured, indicating a controlled or partially-controlled operation with Clifford phases and Hadamard-like mixing.
Accordingly, we express it using a small Clifford+$T$ pattern: phase corrections using $S=T^2$ and a controlled mixing implemented via CNOT and Hadamards.

\subsection{Baseline OpenQASM (compact exact-form template)}
\begin{lstlisting}
// task09_structured_unitary_2.qasm (compact template)
t q[1];
t q[1];

h q[1];
cx q[0], q[1];
h q[1];

tdg q[1];
tdg q[1];
\end{lstlisting}


\section{Discussion}
Tasks 1, 3, 5, and 8 admit exact and compact Clifford+$T$ realizations driven by algebraic structure:
controlled-phase identities, parity-phase gadgets, and the SWAP identity (Eq.~\eqref{eq:swap-identity}).
Tasks 2, 4, and 6 involve non-Clifford angles and/or non-commuting Hamiltonian terms, and thus require approximate synthesis.
The baseline circuits here are intended as correct templates; further reductions in operator-norm distance and $T$-count can be achieved by dedicated
Clifford+$T$ synthesis and optimization toolchains.

\section{Results}

We report the submission metrics for Challenges 1--6 and 8--9 as evaluated by the iQuHACK Superquantum challenge backend.
Each submission is assessed using the total $T$-count and the operator norm distance from the target unitary (up to global phase).
Lower $T$-count corresponds to reduced fault-tolerant cost, while smaller operator norm distance indicates higher accuracy.

\begin{table}[h]
\centering
\caption{Submission results for compiled Clifford+$T$ circuits.}
\label{tab:results}
\begin{tabular}{ccc}
\toprule
\textbf{Challenge} & \textbf{$T$-Count} & \textbf{Operator Norm Distance} \\
\midrule
Challenge 1 & 4 & 1.7320508075688772 \\
Challenge 2 & 2 & 0.22392895220661563 \\
Challenge 3 & 4 & 1.0640641530306731 \\
Challenge 4 & 4 & 1.4142135623730947 \\
Challenge 5 & 0 & $2.46\times 10^{-16}$ \\
Challenge 6 & 3 & 1.57723909841614 \\
Challenge 8 & 3 & 2.0 \\
Challenge 9 & 4 & 3.0154647151211726 \\
\bottomrule
\end{tabular}
\end{table}

Several observations are immediate. Challenge~5 admits an exact Clifford-only implementation corresponding to the $\mathrm{SWAP}$ gate, yielding zero $T$-count and numerical precision-level error.
Challenges~1, 3, 4, 6, 8, and~9 exhibit moderate operator norm distances consistent with coarse but low-cost approximations of non-Clifford rotations.
In particular, Challenges~2 and~6 demonstrate that acceptable accuracy can be achieved with $T$-counts as low as two or three when leveraging problem structure.
Overall, these results highlight the trade-off between approximation accuracy and fault-tolerant resource cost inherent to Clifford+$T$ compilation.


\section{Conclusion}
We presented reusable compilation patterns for several few-qubit unitaries under the restricted gate set $\{H,T,T^\dagger,\mathrm{CNOT}\}$.
Exact constructions were given where possible, and baseline approximations were provided for Hamiltonian exponentials requiring non-Clifford rotations.
These constructions serve as a foundation for iterative optimization toward lower $T$-count while maintaining acceptable approximation error.

\section*{Acknowledgements}
We thank the iQuHACK organizers and Superquantum for releasing the challenge specification and tooling references.



\begin{thebibliography}{9}

\bibitem{iquhack}
iQuHACK, \emph{Official website}. \url{https://iquhack.mit.edu/}

\bibitem{superquantum}
Superquantum, \emph{Official website}. \url{https://superquantum.io/}

\bibitem{rmsynth}
Superquantum, \emph{rmsynth (GitHub repository)}. \url{https://github.com/super-quantum/rmsynth}

\bibitem{qiskit}
IBM Quantum, \emph{Qiskit documentation}. \url{https://qiskit.org/documentation/}

\bibitem{selinger_ross}
N.~J. Ross and P.~Selinger, ``Optimal ancilla-free Clifford+$T$ approximation of $z$-rotations,'' \emph{arXiv:1403.2975} (2014).
\url{https://arxiv.org/abs/1403.2975}

\bibitem{newsynth}
P.~Selinger and N.~J. Ross, \emph{newsynth software}. \url{https://www.mathstat.dal.ca/~selinger/newsynth/}

\bibitem{amy_mosca}
M.~Amy and M.~Mosca, ``$T$-count optimization and Reed--Muller codes,'' \emph{IEEE Transactions on Information Theory} 65, 4771--4784 (2019).

\end{thebibliography}

\end{document}

